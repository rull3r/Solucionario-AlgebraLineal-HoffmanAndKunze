\documentclass[10pt,a4paper]{jhwhw}
\usepackage[utf8]{inputenc}
%Paquetes Necesarios
\usepackage{amsmath}
\usepackage{amsfonts}
\usepackage{amssymb}
\usepackage{makeidx}
\usepackage[spanish,es-lcroman]{babel}
\usepackage{titling}
\usepackage{amsthm}
\usepackage{enumerate}
\usepackage{tikz}
\usepackage{latexsym}
\usepackage{cite}
\usepackage{titlesec}
\usepackage{fancybox}
\usepackage{xparse}
%Quitar el identado de todos los parrafos
\setlength{\parindent}{0cm}
%Para agregar el identado en cada item de enumerate o cualquier otro, usar [\hspace{1cm}(a)]

%Comandos de Letras
\newcommand{\R}{\mathbb{R}}
\newcommand{\N}{\mathbb{N}}
\newcommand{\Z}{\mathbb{Z}}
\newcommand{\Q}{\mathbb{Q}}
\newcommand{\C}{\mathbb{C}}

%Marca de agua en el documento
\usepackage{draftwatermark}
\SetWatermarkText{\textsc{\href{https://rull3r.github.io/}{Visitame en MateTips}}} % por defecto Draft 
\SetWatermarkScale{1} % para que cubra toda la página
%\SetWatermarkColor[rgb]{1,0,0} % por defecto gris claro
\SetWatermarkAngle{55} % respecto a la horizontal

\author{Autor: \href{https://www.facebook.com/ruller}{Raúl García}\\Pagina Web: \href{https://rull3r.github.io/}{MateTips}\\Correo: rull3r@hotmail.com}
\date{Venezuela\\ \today \\}
\title{Solucionario \\\href{https://books.google.co.ve/books?id=XPcoPwAACAAJ}{Álgebra Lineal - Hoffman and Kunze}\\}

\makeindex

\begin{document}
	
	\problema{ }\label{pro:3}
	Prueba los siguientes sistemas de ecuaciones como en el Ejercicio 2.
	\begin{eqnarray*}
		-x_1+x_2+4x_3=0 & x_1-x_3=0\\
		x_1+3x_2+8x_3=0 & x_2+3x_3=0\\
		\frac{1}{2}x_1+x_2+\frac{5}{2}x_3=0 & 
	\end{eqnarray*}

	\solution
	
	\begin{itemize}
		\item Para la ecuación 1 del sistema 1 hacemos $-x_1+x_2+4x_3=a(x_1-x_3)+b(x_2+3x_3)$
		\begin{align*}
		\begin{aligned}
		-1&= a \\
		1&=b	\\
		4&=-a+3b
		\end{aligned}
		\quad
		\Rightarrow
		\quad
		\begin{aligned}
		a&= -1 \\
		b&= 1
		\end{aligned}
		\end{align*}
		
		de modo que $-x_1+x_2+4x_3=-(x_1-x_3)+(x_2+3x_3)$
		
		\item Para la ecuación 2 del sistema 1 hacemos $x_1+3x_2+8x_3=a(x_1-x_3)+b(x_2+3x_3)$
		\begin{align*}
		\begin{aligned}
		1&= a \\
		3&= b	\\
		8&= -a+3b
		\end{aligned}
		\quad
		\Rightarrow
		\quad
		\begin{aligned}
		a&=1 \\
		b&= 3
		\end{aligned}
		\end{align*}
		de modo que $x_1+3x_2+8x_3=(x_1-x_3)+3(x_2+3x_3)$
		
		\item Para la ecuación 3 del sistema 1 hacemos $\frac{1}{2}x_1+x_2+\frac{5}{2}x_3=a(x_1-x_3)+b(x_2+3x_3)$
		\begin{align*}
		\begin{aligned}
		\frac{1}{2}&= a \\
		1&= b	\\
		\frac{5}{2}&= -a+3b
		\end{aligned}
		\quad
		\Rightarrow
		\quad
		\begin{aligned}
		a&=\frac{1}{2} \\
		b&= 1
		\end{aligned}
		\end{align*}
		de modo que $\frac{1}{2}x_1+x_2+\frac{5}{2}x_3=\frac{1}{2}(x_1-x_3)+(x_2+3x_3)$
		
		\item Para la ecuación 1 del sistema 2 hacemos $x_1-x_3=a(-x_1+x_2+4x_3)+b(x_1+3x_2+8x_3)+c(\frac{1}{2}x_1+x_2+\frac{5}{2}x_3)$
		\begin{align*}
		\begin{aligned}
		1&= -a+b+\frac{c}{2} \\
		0 &=	a+3b+c	\\
		-1&=4a+8b+\frac{5}{2}c
		\end{aligned}
		\quad
		\Rightarrow
		\quad
		\begin{aligned}
		a&=-\frac{3}{4} \\
		b&=\frac{1}{4}	\\
		c&=0
		\end{aligned}
		\end{align*}
		de modo que $x_1-x_3=-\frac{3}{4}(-x_1+x_2+4x_3)+\frac{1}{4}(x_1+3x_2+8x_3)+c(\frac{1}{2}x_1+x_2+\frac{5}{2}x_3)$  $\Rightarrow$  $x_1-x_3=-\frac{3}{4}(-x_1+x_2+4x_3)+\frac{1}{4}(x_1+3x_2+8x_3)$
		
		
		\item Para la ecuación 2 del sistema 2 hacemos $x_2+3x_3=a(-x_1+x_2+4x_3)+b(x_1+3x_2+8x_3)+c(\frac{1}{2}x_1+x_2+\frac{5}{2}x_3)$
		\begin{align*}
		\begin{aligned}
		0&= -a+b+\frac{c}{2} \\
		1 &=	a+3b+c	\\
		1&=4a+8b+\frac{5}{2}c
		\end{aligned}
		\quad
		\Rightarrow
		\quad
		\begin{aligned}
		a&=\frac{1}{4} \\
		b&=\frac{1}{4}	\\
		c&=0
		\end{aligned}
		\end{align*}
		de modo que $x_2+3x_3=\frac{1}{4}(-x_1+x_2+4x_3)+\frac{1}{4}(x_1+3x_2+8x_3)+0(\frac{1}{2}x_1+x_2+\frac{5}{2}x_3)$ $\Rightarrow$  $x_2+3x_3=\frac{1}{4}(-x_1+x_2+4x_3)+\frac{1}{4}(x_1+3x_2+8x_3)$
	\end{itemize}
	
\end{document}